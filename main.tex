\documentclass{article}
\usepackage{graphicx} % Required for inserting images
\usepackage{amsmath} 

\title{Lab 5}
\author{m.carvajalp - 202014203}
\date{2025}

\begin{document}

\maketitle

\section{Problema 1}

\subsection{Datos del problema}
Se desea analizar la función polinómica cúbica:
\[
f(x) = 3x^3 - 10x^2 - 56x + 50,
\]
definida en el intervalo:
\[
x \in [-6,\,6].
\]

\subsection{Objetivo}
El objetivo consiste en encontrar los \textbf{puntos críticos} \(x^*\) tales que el gradiente (en una dimensión, la derivada) se anule:
\[
f'(x^*) = 0,
\]
y posteriormente \textbf{clasificar} dichos puntos como mínimos o máximos locales a partir del signo de la segunda derivada \(f''(x^*)\).

\subsection{Derivadas analíticas}
La primera y segunda derivada de \(f(x)\) son:
\[
f'(x) = 9x^2 - 20x - 56,
\]
\[
f''(x) = 18x - 20.
\]

\subsection{Condiciones de optimalidad}
El punto crítico \(x^*\) debe satisfacer:
\[
f'(x^*) = 0,
\]
y se clasifica según el criterio de la segunda derivada:
\[
\begin{cases}
f''(x^*) > 0 & \Rightarrow \text{Mínimo local},\\[4pt]
f''(x^*) < 0 & \Rightarrow \text{Máximo local},\\[4pt]
f''(x^*) = 0 & \Rightarrow \text{Indeterminado o punto de inflexión.}
\end{cases}
\]

\subsection{Método de Newton--Raphson para extremos}
Para localizar los puntos donde \(f'(x) = 0\), se emplea el método de Newton--Raphson, cuya iteración general es:
\[
x_{k+1} = x_k - \alpha\,\frac{f'(x_k)}{f''(x_k)},
\]
donde \(0 < \alpha \leq 1\) es un \textit{factor de amortiguación} que mejora la estabilidad numérica cuando el punto inicial se encuentra lejos del óptimo.

\subsection{Criterios de parada}
El método finaliza cuando:
\[
|f'(x_k)| < \varepsilon,
\]
o cuando se alcanza el número máximo de iteraciones \(N_{\text{max}}\).
Solo se aceptan soluciones \(x_k \in [-6,\,6]\), y cada punto crítico se evalúa junto con su valor de función \(f(x^*)\) y su clasificación según \(f''(x^*)\).


\subsection{Descripción de la Implementación}

La implementación del método de Newton--Raphson para encontrar extremos locales
se desarrolló completamente desde cero, siguiendo los lineamientos del laboratorio.
Se emplearon únicamente las bibliotecas permitidas: \texttt{NumPy}, \texttt{SymPy} y \texttt{Matplotlib}.

El programa se organiza en cinco secciones principales:

\begin{enumerate}
    \item \textbf{Definición simbólica de la función:} se declara la variable simbólica \(x\) y se define la función objetivo
    \(f(x)=3x^3-10x^2-56x+50\). A través de \texttt{SymPy} se calculan analíticamente \(f'(x)\) y \(f''(x)\),
    que luego se convierten en funciones numéricas con \texttt{lambdify} para evaluación vectorizada.
    
    \item \textbf{Implementación del método de Newton--Raphson:} se desarrolla la función
    \texttt{newton\_extremo\_1d()}, que aplica la iteración
    \[
    x_{k+1}=x_k-\alpha\,\frac{f'(x_k)}{f''(x_k)},
    \]
    donde \(0<\alpha\leq1\) es un factor de amortiguación.  
    El proceso se repite hasta que \(|f'(x_k)|<\varepsilon\) o se alcanza el número máximo de iteraciones \(N_{\text{max}}\).
    
    \item \textbf{Clasificación de puntos críticos:} mediante la función \texttt{clasificar\_extremo()},
    se determina el tipo de extremo usando el signo de la segunda derivada:
    \[
    \begin{cases}
    f''(x^*)>0 & \Rightarrow \text{mínimo local},\\[4pt]
    f''(x^*)<0 & \Rightarrow \text{máximo local}.
    \end{cases}
    \]
    
    \item \textbf{Barrido de semillas iniciales:} se ejecuta Newton--Raphson desde múltiples puntos
    equiespaciados en el intervalo \([-6,6]\) para asegurar la detección de todos los extremos.
    Las soluciones convergentes se agrupan por cercanía y se reportan como puntos críticos únicos.
    
    \item \textbf{Visualización y análisis:} se generan dos gráficas:
    \begin{itemize}
        \item la función \(f(x)\) con los puntos mínimos y máximos marcados, y  
        \item la evolución de las trayectorias \(x_k\) para cada semilla, lo que permite observar la convergencia.
    \end{itemize}
\end{enumerate}






\end{document}
